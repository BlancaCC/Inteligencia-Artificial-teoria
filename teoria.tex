\documentclass[a4paper]{article}
%\documentclass[12pt,a4paper,oneside,draft]{article}%{report}
\usepackage[utf8]{inputenc}
\usepackage[spanish]{babel}
\usepackage{amsmath} 
\usepackage{amssymb} % utlizar mathbb


\title{Teoría Inteligencia Artificial}
\author{Blanca Cano Camarero}

\begin{document}
\begin{titlepage}
\maketitle
\tableofcontents

\end{titlepage}




\section{Agentes}
    \textit{Concepto de Agente. Agentes racionales comparados con  agentes inteligentes. Arquitecturas de agentes.}

\subsection{Concepto agente}

Un agente es cualquier cosa capaz de percibir su medio ambiente con la ayuda de sensores 
y actuar en ese medio utilizando actuadores (Stuar Russell,Inteligencia Artificial. Un enfoque moderno, 2011, p. 37).
Entendiendo por actuador al elemento que reacciona a un estímulo. 
mediante una acción. La secuencia de percepciones de un agente refleja el historial 
completo de lo que el agente ha recibido. Un agente tomará una decisión en un momento dado
 dependiendo de la secuencia completa 
o de una parte de ella. En términos matemáticos se puede decir que el comportamiento del 
agente viene dado por la función del agente que a cada percepción le asocia una acción. 
Un agente que cumpla lo anterior será autónomo (no necesita al humano) y proactivo (toma sus
 propias decisiones). Además, es deseable que sea social para servirse de otros agentes para
  lograr su objetivo.

Las características indispensables de los agentes son: la capacidad estímulo respuesta, 
un corpontamiento proactivo y autonomía. Otra caracterísitica posible 
sería la de ser social, es decir interactuar con otros agentes o humanos. 



\subsection{ Agentes inteligentes y agentes racionales}  

Se \textbf{consideran inteligentes} a aquellos agentes capaces de superar el test de Turing, es decir si consigue
 aparentar ser un humano para ciertas pruebas. \paragraph{}

 Los \textbf{agentes racionales} son aquellos que buscan el comportamiento mejor dentro de las opciones de las que
 disponen. 

 \subsection{Arquitecturas de los agentes}  
Russel define un agente como la arquitectura que tiene más el programa que utiliza. 

La arquitectura coordina el manejo de la información obtenida por los sensores, ejecución de los programas y actuadores. 

\subsubsection*{Clasificación de arquitecturas por su topología}

\begin{itemize}
  \item \textbf{ Arquitectura horizontal.} Todas las capas que constituyen al agente con capaces
  de percibir el estímulo del entorno y realizar acciones. 

  \item \textbf{Arquitectura vertical.} Una capa percibe los estímulos del entorno, transmite la información a otras capas \textit{superiores}
  y es la última la que realiza las acciones. 

\item \textbf{Arquitectura híbrida.} Se comporta como la arquitectura vertical ya que todas las capas forman una cadena
recibiendo y transmitiendo la información a adistinto niveles, 
pero es la capa que percibe el entorno la que se encarga de realizar las acciones.

\end{itemize}

\subsubsection*{Clasificación de arquitecturas por su abstracción.}
\begin{itemize}
  \item \textbf{Arquitecturas deliberativas.} Comprenden un sistema simbólico de la realidad, hipótesis sobre estos y un agente deliberativo 
  cuyas decisiones se realizan a través de un razonamiento lógico basado en emparejamiento de patrones y manipulaciones simbólicas.
  Suelen estar diseñados con una estructura vertical y deben tener en cuenta el tiempo de respuesta y manejo de símbolos apropiado. 
  
  \item \textbf{Arquitecturas reactiva.} Aquellas que no incluyen ninguna clase de modelo centralizaod de representación
  simbólica del mundo, y no hace uso del razonamiento complejo; se construyen bajo esquemas de percepción y acción, con lo que se busca
  una respuesta rápida y eficaz. Aunque existen agentes reactivos con memoria con el fin de conseguir retroalimentación.  
  
  \item \textbf{Arquitecturas híbridas. } Combinan arquitecturas deliberativas y reactivas, un paradigma recomendado para la respuesta a problemas
  no previstos durante el desarrollo del plan. 
\end{itemize}


\newpage
\section{ Características de los Agentes reactivos y deliberativos.}
 \textit{Características de los Agentes reactivos y deliberativos. Similitudes y diferencias. Arquitecturas.}

\subsection{Agentes reactivos}
Son aquellos agentes que siguen un ciclo acción-reacción y no poseen ningún modelo del mundo, aunque si posee memoria puede ir
construyéndolo. Su comportamiento está basado en reaccionar a sensores, la cual puede modifificarse si existe retroalimentación, 
por información almacenada en memoria.

Es necesario para su buen funcionamiento que el diseño abarque todos los estados posibles. 
Sus características principales es que realizan pocos cálculos, almacenan todo en memoria y usan arquitecturas 
horizontales. 

Un ejemplo de estrategias de diseño es: 

\begin{itemize}
  \item  \textbf{Arquitectura de subsunción:} Consiste en agrupar comportamientos en módulos y dada una percepción, uno de los comportamientos se activará. (Pensar en un switch).
  \item  \textbf{Agentes reactivos con memoria.} Mejorar la precisión de respuesta gracias a la memoria de estímulos recibidos. 
  Un ejemplo de este es la implementación basada en pizarras, es decir una memoria común a todos los programas del agente (módulos de conocimiento).
\end{itemize}

\subsection{Agentes deliberativos}

Los agentes deliberativos son aquellos que tienen un modelo del mundo y elboran un plan según los efectos de
su acciones sobre esos símbolos físicos. Sus decisiones se realizan sobre esos símbolos físicos, a partir de las hipótesis que poseen,
no reaccionan a los eventos. 
Usan arquitecturas verticales.

A la hora de diseñar, la abstracción de la realidad para desarrollar el modelo  o estados del agente, es donde reside la dificultad, 
,es más, si se usa una arquitectura deliberativa pura y el sistema lógico no es completo podría no alcanzar nunca una solución (((((y si es completo no será consistente xD)))))). 



\subsection*{Conclusión}
Las diferencia esenciales son. 

\subsubsection*{agentes reacitvos: }

\begin{itemize}
  \item Diseño explícito y completo de situaciones posibles.
  \item Pocos cálculos.
  \item Almacenan todo en memoria.
  \item Arquitecturas horzontales.
\end{itemize}

\subsubsection*{Los agentes deliberativos: }

\begin{itemize}
  \item Razonamiento lógico sobre modelos (conocen el mundo).
  \item Usan arquitecturas verticales.
\end{itemize}

\newpage

\section{Preguntas pendientes del tema 1}
3.- Describir brevemente los métodos de búsqueda no informada.

4.- El concepto de heurística. Como se construyen las heurísticas. Uso de las heurísticas en IA.

5.- Los métodos de escalada. Caracterización general. Variantes.

6.- Características esenciales de los métodos “primero el mejor”.

7.- Elementos esenciales del algoritmo A*.

8.- Elementos esenciales de un algoritmo genético.

\section{Preguntas pendientes del tema 2}

\section{Componentes de un juego}
\subsection{Definición de juego}

Un juego es cualquier situación de decisión de varios agentes, (jugadores)
gobernada por un conjunto de reglas y con un resultado bien definido, caracterizada porque ninguno de los 
jugadores con su sola acutación puede determinal el resutlado (independecia estratégica). 

Es decir, nos encontramos en un entorno multiagente, donde la imprevisibilidad de estos introducen 
contingencias. En IA los juego son por lo general los de suma cero (la ganancio o pérdida de un jugador se equilibra con la del otro)
, de dos jugadores, por turnos, deterministias,
de información perfecta. 

Las compentes de un juego son: 

\begin{itemize}
  \item \textbf{Número de jugadores.}
  \item Si es \textbf{de información perfecta}, toda la información del juego es conocida por todos los jugadores (vg ajedrez). 
  o por el contrario \textbf{ de información imperfecta} se oculta la información parcialmente (póker).
  \item \textbf{Determinista o no}, si el azar influye (la oca).
  \item  Si es \textbf{Por turnos}
  \item Existencia o no de pagos colaterales. Que la acción de un jugador, beneficie a otro llegando así a un equilibrio de Nash. 
  \item Juegos de \textbf{suma nula} el beneficio total para todo los jugadores suma 0, en contraposición de \textbf{suma no nula}, 
  donde la ganancia de un jugador no necesariamente significa la pérdida del otro.  
\end{itemize}

\subsection{Equilibrio de Nash}
(Wikipedia principal fuente)

Es un concepto de solución para juegos multijugadores del cual se toma como hipótesis 
que todo jugador a adoptado su mejor estrategia y todos conocen la estrategia de los otros.

De aquí deducimos de que tras esto no beneficia cambiar de estrategia ya que el resto no la va a alterar
y en tal cicunstancia ya teniamos la mejor. 

Además el resultado maximiza el beneficio individual no el de conjunto, lo cual podría dar a que 
si todos coordinaran este fuera mejor para todos (en un juego de dos jugadores esto no tiene sentido, pero sí en "alianzas"). 


Un ejemplo sería el juego del dilema del prisionero, pero introduciendo más jugadores, 
donde la mejor estrategia individual sería declarar ya que minimiza la pérdida, sin embargo 
si todos cooperaran no confesando el beneficio global sería mayor. 

\newpage

\section{ramificación}
\textit{Qué es el factor de ramificación y cómo afecta a la complejidad de un juego? Describe en líneas generales el algoritmo minimax y el de la poda alfa-beta }
(Capítulo 6 del Russell Norvig)

\subsection{Definición factor de ramificación}
Dada una estructura de árbol el \textbf{factor de ramificación} es el número medio de hijos en cada nodo. 
Que esto en los juegos se manifiesta como el número de movimientos posibles por turno.  

Cuanto mayor sea el factor de ramificación, mayor será la complejidad a la hora de buscar estrategias, ya que 
llegado cierto punto es imposible una búsqueda exhaustiva de la mejor solución.

En ajedrez el 
factor de remificación es 35, que por combinatoria en $m$ movimientos se habrá habido una media de $35^m$ posibilidades.  
Este es el mismo motivo por el que solo se ha conseguido una IA que juegue como un aficionado en el GO.

De aquí surgen la estrategias de poda de nodos: 

\subsection{Algoritmo min-max}



\newpage
3.- ¿Que problemas plantea el cálculo de predicados en la resolución de problemas de IA?
4.- Modelos de conocimiento heredable ¿Qué tipo de conocimiento organizan las redes semánticas? Describir en líneas generales el concepto de “frame”.
5.-  Estructura y componentes de un sistema experto 
6.- Paradigmas de Aprendizaje Automático.
7.- Describir el problema del ruido y el del sobreajuste en aprendizaje automático.
8.- ¿Qué son y como se construyen los arboles de decisión

\end{document}
